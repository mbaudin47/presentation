% Copyright (C) 2012 - EDF R&D - Michael Baudin

% To highlight source code
\usepackage{listings}
\definecolor{darkgreen}{rgb}{0,0.5,0}
\definecolor{violet}{rgb}{0.5,0,1}

% \usepackage{lmodern}% http://ctan.org/pkg/lm

\usetheme{Darmstadt} % http://tex.stackexchange.com/questions/177042/beamer-latex-customized-formats

\useoutertheme[subsection=false,footline=authortitle]{miniframes}
% RGB scaled on 0-255 scale (section 17.1.1), colors pulled from title block
\usecolortheme[RGB={44, 131, 82}]{structure}

% hide header:
\setbeamertemplate{headline}{}
\setbeamertemplate{navigation symbols}{}

\usepackage[utf8]{inputenc}
\usepackage[T1]{fontenc}

%\usepackage[french]{babel}
%\uselanguage{French}
%\languagepath{French}

\def\bx{{\bf x}}
\def\RR{\mathbb{R}}

% The physical model
\newcommand{\model}{g}
\newcommand{\inputDim}{d}  % The input dimension of the model
\newcommand{\outputDim}{{d_Y}}  % The output dimension of the model
\newcommand{\metaModel}{\widetilde{\model}}  % The surrogate model

% Probabilistic modeling
\newcommand{\inputRV}{\vect{X}}  % The input random vector of the model
\newcommand{\outputRV}{\vect{Y}}  % The output random vector of the model
\newcommand{\inputMeasure}{\mu_{\inputRV}}  % The distribution of the input random vector of the model
\newcommand{\outputMeasure}{\mu_{\outputRV}}  % The distribution of the output random vector of the model
\newcommand{\standardRV}{\vect{Z}}  % The standard random vector (e.g. for polynomial chaos expansion)
\newcommand{\RVU}{\vect{U}}

% For matrixes and vectors
\newcommand{\vect}[1]{{\mathbf{\boldsymbol{{#1}}}}}
\newcommand{\mat}[1]{{\vect{\vect{#1}}}}
\newcommand{\Tr}[1]{{#1}^\top}

% Probabilistic operators
\newcommand{\muX}{\vect{\mu}_{\:X}}
\newcommand{\Var}[1]{{\operatorname{Var}}\left(#1 \right)}
\newcommand{\Cov}[1]{{\operatorname{Cov}}\left(#1 \right)}
\newcommand{\Expect}[1]{{\mathbb{E}}\left[ #1 \right]}
\newcommand{\Econd}[2]{{\mathbb{E}}_{#1}\left[ #2 \right]}
\newcommand{\Prob}[1]{{\mathbb{P}}\left(#1 \right)}
\newcommand{\ProbCond}[2]{{\mathbb{P}}_{#1}\left(#2 \right)}
\newcommand{\ech}{\left\{ x_1, \, \dots\,, x_N  \right\}}
\newcommand{\matcov} {\mathbf{C}}
\newcommand{\matcor} {\mathbf{R}}
\newcommand{\fcar}[2] {{\mathbf{1}}_{#1}(#2)}

\newcommand{\pyvar}[1]{\texttt{#1}}

\def \ot {OpenTURNS}

\hypersetup{colorlinks=true}

\lstset{
  % general command to set parameter(s)
   %basicstyle=\footnotesize\ttfamily, %
   %basicstyle=\normalsize \ttfamily, %
   basicstyle=\scriptsize\ttfamily, %
   keywordstyle=\color{violet}\bfseries,%
   commentstyle=\color{darkgreen}\bfseries,%
   showspaces=false,%
   stringstyle=\color{red}\bfseries, 
   otherkeywords={Gumbel, TruncatedDistribution, LatentVariableModel, %
        SmoothedUniformFactory, Uniform, PythonFunction, JointDistribution, %
        StudentCopula, StudentCopulaFactory, RankSobolSensitivityAlgorithm, %
	   MonteCarlo, IntervalMesher, PointToFieldFunctionalChaosAlgorithm, FieldFunctionalChaosSobolIndices, HistogramFactory, %
	   Graph, BoxCoxFactory, CompositeProcess, WhittleFactory, ARMAFactory, %
	   Basis, TrendFactory, BoxCoxTransform, SpatialFunction, TimeSeries, %
	   WelchFactory, GreaterOrEqual, IndependentCopula, %
	   FunctionalChaosAlgorithm, LowDiscrepancyExperiment, HaltonSequence, LOLAVoronoi,
	   SystemFORM, UnionEvent, IntersectionEvent, 
     PointToField FunctionalChaosAlgorithm},
}


\usepackage{adjustbox}
\usepackage[normalem]{ulem}
